% !TeX encoding = UTF-8
% !TeX program = pdflatex
% !BIB program = biber

\documentclass[]{lni}


% TODO die hier vlt Book capters and books rausnehmen sind die (10409934) und (angwin2022machine)

%% Schöne Tabellen mittels \toprule, \midrule, \bottomrule
\usepackage{booktabs}

%% Zu Demonstrationszwecken
\usepackage[]{blindtext}


%%Für bib
\usepackage[backend=biber, style=ieee]{biblatex}
\AtBeginBibliography{\selectlanguage{english}}



\addbibresource{education.bib} % Hinzufügen der .bib-Datei

\usepackage[english]{babel}

\begin{document}

\title[]{Education for Sustainable Development: The Role of Artificial Intelligence}

\author[1]{Nicolas Zander}{nicolas.zander@st.oth-regensburg.de}{}
\affil[1]{OTH Regensburg\\Seybothstraße 2, Regensburg\\Matrikelnummer: 3363036}

\maketitle

\begin{abstract}

This paper explores how artificial intelligence (AI) can contribute to specific Sustainable Development Goals (SDGs) set by the United Nations, 
particularly those related to quality education and reducing global inequalities, through the lens of education for sustainable development (ESD).
An analysis of current research on the integration of AI driven tools in education shows that these technologies can personalise the learning experience for learners. 
This helps students to achieve their best academic performance. However, several challenges have been highlighted,
such as unequal access to AI tools, algorithmic bias, privacy issues and an over reliance on technology that could undermine the importance of human interaction in learning. 
These findings highlight the ambivalent nature of AI in education, revealing both the potential for unprecedented advances and the critical need for ethical considerations and
equitable access. This approach ensures that the benefits of AI in education are shared by all learners, regardless of background or culture.
\end{abstract}

\begin{keywords}
Education for Sustainable Development \and AI-driven learning \and Personalised adaptive learning \and artificial intelligence in education 
\end{keywords}


%%% Beginn des Artikeltexts
\section{Introduction}
%
A considerable number of studies have addressed the question of how AI can be employed to facilitate a more sustainable utilization of resources. However, 
this research primarily focuses on areas such as the construction industry, logistics, the water segment, Industry 4.0, and healthcare 
\cite{KAR2022134120}.
Nevertheless, education is also a critical resource that must be managed sustainably. It is essential that all learners possess the requisite knowledge and 
skills to promote sustainable development in the future 
\cite{OFlaherty2018}, especially at a time when technological progress is occurring at an accelerated pace \cite{Judijanto2022}.
This is also reflected in the UN's 2030 Sustainable Development Goals, which aim to promote inclusive, equitable, and quality education and provide 
lifelong learning opportunities for all
\cite{un_sdg4}. 
Therefore the digitization of educational resources as part of current education reforms can be a catalyst for AI in education. \cite{KAMYAB2023101566} \cite{cai2021review}.
However the introduction of AI into educational processes raises a number of questions that must be addressed in order to ensure the 
sustainable and effective implementation of this technology.
Given the pivotal role of education for sustainable development, it is imperative to ascertain the potential of AI in ensuring sustainable education in the future.
%
% das sind 1200 zeichen
%
% ! Methoden 
\section{Methods}
In order to collect pertinent information from reliable sources, we employed a review process that included the use of specific criteria. Accordingly, the following criteria for inclusion are presented:
\begin{itemize}
    \hangindent=0.5in
    \hangafter=1
    \item Articles written in English
    \item Articles published between 2015 and 2024
    \item Articles that appeared in conference proceedings and scholarly journals
    \item Books and Book Chapters
\end{itemize}

In light of the aforementioned objectives, the following research questions were developed with the intention of enabling us to obtain information from primary studies:
\begin{itemize}
    \hangindent=0.5in
    \hangafter=1
    \item What are the potential benefits of AI-driven technologies for enhancing the effectiveness and accessibility of education?
    \item What are the key challenges that need to be addressed in order to achieve this goal?
\end{itemize}
%
%
%
% ? Hauptteil
\section{Results}
%
\subsection{Benefits}
%
When considering the education sector, it is evident that there is a clear trend towards the digitization of learning materials. 
This is accompanied by the emergence of a multitude of online courses and platforms, which generate a significant amount of data \cite{Lin2023}. The analysis of 
such datasets can be conducted using AI algorithms to identify complex patterns or to develop new teaching methods and approaches.
%
In this context, AI can facilitate a transformation in traditional learning methodologies by integrating various innovative applications \cite{Judijanto2022}, 
as outlined by Yousuf Mohd \cite{9670009}. 
%
% dieser punkt sind so 1000 zeichen 
One example of this transformation is personalized learning \cite{shemshack2020systematic}, which involves implementing learning materials that can adapt 
to the learner's needs in real time. Through personalized learning, each student receives unique materials tailored to their specific requirements \cite{9670009}. 
These materials can adjust the difficulty level of tasks according to the learner's pace and provide assignments designed to match the learner's type and sensory 
preferences \cite{9457948} \cite{7742268}. This approach addresses the strengths, weaknesses, and interests of students in an effective manner, supporting and motivating them throughout their learning 
process \cite{shemshack2020systematic}. Additionally, it reduces the amount of time and resources required in comparison to traditional learning \cite{Lin2023}.
As a result, students' academic performance improves \cite{shemshack2020systematic} \cite{9457948} and the likelihood of dropout is reduced \cite{9368432}.
By enabling personalized learning, the education system can transition away from the traditional classroom approach towards a more student-centered model \cite{Lin2023}.
%
%
% Glaube das passt hier 
However, the advent of the new student centered approach based on artificial intelligence should not be viewed as a replacement for traditional teachers. Rather, 
the technology can assist teachers by relieving them of some of their tasks \cite{7742268} \cite{Lin2023} \cite{Tanveer2020}. Nevertheless, the integration of 
these new technologies will undoubtedly alter the role of teachers, potentially leading to a transformation in the approach to teaching \cite{Lin2023} \cite{9670009}.  
Furthermore, the technologies can address the shortage of teachers or educational opportunities \cite{cai2021review} \cite{Blanchard2015}.  
%
% Neuer Punkt zu ITS
Another technology that can assist in addressing the growing disparity between the scarcity of qualified specialist educators and the expanding number of learners 
is the utilization of intelligent tutoring systems (ITS) \cite{9670009}. ITS supports educators by providing instruction and immediate feedback to students, effectively 
helping them manage and resolve problems as they arise \cite{9582029}. Therefore, ITS creates a smart learning environment that allows students to learn 
efficiently without the constant presence of a teacher \cite{9582029}. Additionally, in combination with personalized learning the  systems can track student 
progress and adapt to individual learning styles, which enhances the overall learning experience. By leveraging ITS, educational institutions can ensure that all 
students receive high-quality education, regardless of the availability of specialized teaching staff.
%
%  bs hier sind es im findings 1900 zeichen 
%
%
%
\subsection{Challenges}
% ? Challanges ab hier 
One of the most significant obstacles is the absence of the requisite infrastructure to effectively integrate novel AI-driven technologies into the conventional 
pedagogical framework \cite{Lin2023}. This issue is particularly prevalent in developing countries, where access to and affordability of electricity and hardware 
are major barriers \cite{Tanveer2020}. For example, only 31 percent of public schools in Sub-Saharan Africa have access to electricity \cite{Global2021}. Additionally, the 
required hardware and systems can be very expensive.
%
This situation could potentially exacerbate the disparity between affluent and disadvantaged individuals, as the deployment 
of costly technology in education may inadvertently hinder economically challenged students from accessing quality educational opportunities \cite{9670009}.
%
Additionally, there is the challenge of ensuring that students and educators have the necessary skills to navigate the evolving landscape of digital technologies. 
This could necessitate the provision of training for each individual \cite{Tanveer2020} \cite{10409934}.
%
Another issue arises in the context of data protection and security \cite{Lin2023}. The challenge here is that while collecting and processing the personal data 
of learners, security and data protection must be guaranteed \cite{Tanveer2020}.
%
Furthermore, the potential for AI models to contain unintended biases, which may arise from the selection of training data, 
represents a significant challenge \cite{Lin2023}. Such biases could result in algorithms discriminating against minorities, thereby undermining the fundamental 
objectives of ESD \cite{angwin2022machine} \cite{mayfield-etal-2019-equity}.
%
A further consideration is the potential for learners to become overly reliant on technology, which could result in dependency on the technology itself. This could 
ultimately lead to learners being unable to perform without it \cite{Chin2010}.
%
Lastly, another issue that arises with AI in general, and particularly in the field of education, is the interpretability or traceability of the results provided by 
AI \cite{9670009}.
%
%
%
\section{Dicussion} % 950 Zeichen 
The study has demonstrated that the utilization of AI in the educational sector is fundamentally justified and may even result in certain advantages. 
Nevertheless, the study has also identified several challenges associated with the implementation of AI.
%
Although the findings were collected based on articles selected according to the criteria outlined in the Methods section, a systematic review process was not 
employed in this research. A more reliable set of results could have been achieved if the sources had been subjected to a systematic review process, 
such as the one presented by Kitchenham \cite{kitchenham2007guidelines}.
%
Moreover, including a wider range of sources would have enriched the analysis. This would have facilitated a more nuanced understanding of the 
impact of AI in education for sustainable development. By adopting these enhanced methodologies, the study could offer more actionable insights and recommendations 
for educators, policymakers, and researchers working in this evolving field.
%
%
%
%
\section{Conclusions and Future Directions}
In summary, AI has the potential to significantly transform and improve our education system. By personalizing learning content, it can enhance student performance \cite{shemshack2020systematic} \cite{9457948}. 
Additionally, AI can support teachers in their educational tasks by taking over some of their workload. This can help address the teacher shortage that many education systems in developing countries \cite{pholphirul2023teacher} 
as well as industrialized countries \cite{porsch2023teacher} are struggling with. Intelligent Tutoring Systems (ITS) contribute to this effort by providing personalized instruction and immediate feedback to students, 
thus ensuring a high-quality learning experience even in the absence of specialized teaching staff. \cite{9670009}.

However, essential challenges must be overcome to ensure the sustainable and effective implementation of these technologies. In particular, infrastructural barriers need to be addressed, 
and issues of data security and privacy must be resolved \cite{Tanveer2020} \cite{Lin2023}. Moreover, it is crucial to enhance the digital skills of teachers and students and to minimize 
potential biases in AI models that could lead to the disadvantage of certain individuals \cite{angwin2022machine}.

Ultimately, with the right approach, AI can make a decisive contribution to sustainable education worldwide, although there are still some open questions and issues to be resolved. 
Future research should focus on developing scalable AI solutions that can be easily integrated into existing educational frameworks. Policymakers should also consider creating supportive 
policies and funding mechanisms to address infrastructural challenges and promote the adoption of AI in education. 

\newpage 



\printbibliography % im Falle der Nutzung von biblatex
\end{document}



