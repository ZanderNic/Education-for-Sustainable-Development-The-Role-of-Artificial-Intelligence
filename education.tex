% !TeX encoding = UTF-8
% !TeX program = pdflatex
% !BIB program = biber

\documentclass[]{lni}


%% Schöne Tabellen mittels \toprule, \midrule, \bottomrule
\usepackage{booktabs}

%% Zu Demonstrationszwecken
\usepackage[]{blindtext}


%%Für bib
\usepackage[backend=biber, style=ieee]{biblatex}

\addbibresource{education.bib} % Hinzufügen der .bib-Datei


\begin{document}

\title[]{Education for Sustainable Development: The Role of Artificial Intelligence}

\author[1]{Nicolas Zander}{nicolas.zander@st.oth-regensburg.de}{}
\affil[1]{OTH Regensburg\\Seybothstraße 2, Regensburg\\Matrickelnummer: 3363036}

\maketitle

\begin{abstract}

This paper explores how artificial intelligence (AI) can contribute to specific Sustainable Development Goals (SDGs) set by the United Nations, 
particularly those related to quality education and reducing global inequalities, through the lens of education for sustainable development (ESD).
An analysis of current research on the integration of AI-driven tools in education shows that these technologies can personalise the learning experience for learners. 
This helps students to achieve their best academic performance. However, several challenges have been highlighted,
such as unequal access to AI tools, algorithmic bias, privacy issues and an over-reliance on technology that could undermine the importance of human interaction in learning. 
These findings highlight the ambivalent nature of AI in education, revealing both the potential for unprecedented advances and the critical need for ethical considerations and
equitable access. This approach ensures that the benefits of AI in education are shared by all learners, regardless of background or culture.
\end{abstract}

\begin{keywords}
Education for Sustainable Development \and AI-driven learning \and Personalised adaptive learning \and artificial intelligence in education 
\end{keywords}


%%% Beginn des Artikeltexts
\section{Introduction}
%
A considerable number of studies have addressed the question of how AI can be employed to facilitate a more sustainable utilization of resources. However, 
this research primarily focuses on areas such as the construction industry, logistics, the water segment, Industry 4.0, and healthcare 
\cite{KAR2022134120}.
Nevertheless, education is also a critical resource that must be managed sustainably. It is essential that all learners possess the requisite knowledge and 
skills to promote sustainable development in the future 
\cite{OFlaherty2018},
 especially at a time when technological progress is occurring at an accelerated pace \cite{Judijanto2022}.
This is also reflected in the UN's 2030 Sustainable Development Goals, which aim to promote inclusive, equitable, and quality education and provide 
lifelong learning opportunities for all
\cite{un_sdg4}. 
Therefore the digitization of educational resources as part of current education reforms can be a catalyst for AI in education. \cite{KAMYAB2023101566} \cite{cai2021review}.
However the introduction of AI into educational processes raises a number of questions that must be addressed in order to ensure the 
sustainable and effective implementation of this technology.
Given the pivotal role of education for sustainable development, it is imperative to ascertain the potential of AI in ensuring sustainable education in the future.
%
%
%
% ! Methoden 
\section{Methods}
In order to collect pertinent information from reliable sources, we employed a review process that included the use of specific criteria. 
Accordingly, the following criteria for inclusion are presented:\\
- Articles written in English \\
- Articles published between 2015 and 2024\\
- Articles that appeared in conference proceedings and scholarly journals \\
- Books and Book Chapters \\ % TODO die hier vlt rausnehmen sind die (10409934) und (angwin2022machine)

In light of the aforementioned objectives, the following research questions were developed with the intention of enabling us to obtain information from primary studies: \\
- What are the potential benefits of AI-driven technologies for enhancing the effectiveness and accessibility of education for sustainable development?\\ 
- What are the key challenges that need to be addressed in order to achieve this goal?
%
%
%
% ? Hauptteil
\section{Findings}
When considering the education sector, it is evident that there is a clear trend towards the digitization of learning materials. 
This is accompanied by the emergence of a multitude of online courses and platforms, which generate a significant amount of data \cite{Lin2023}. The analysis of 
such datasets can be conducted using AI algorithms to identify complex patterns or to develop new teaching methods and approaches.
% Yousuf, Mohd Applications in Education
In this context, AI can facilitate a transformation in traditional learning methodologies by integrating various innovative applications \cite{Judijanto2022}, 
as outlined by Yousuf Mohd in his article \cite{9670009}. 
%
One illustrative example of this transformation is personalized learning \cite{shemshack2020systematic}, which involves the implementation of learning materials 
that can adapt to the learner's needs in real time. Through personalized learning, each student receives unique learning materials tailored to their specific needs 
\cite{9670009}, which can include adjusting the difficulty level of tasks and providing assignments designed according to the learner's type and sensory preferences
 \cite{9457948} \cite{7742268}. This approach effectively supports and motivates students throughout their learning process \cite{shemshack2020systematic}, ultimately 
leading to improved academic performance \cite{shemshack2020systematic} \cite{9457948}.
By enabling personalized learning, the education system can transition away from the traditional classroom approach towards a more student-centered approach 
\cite{Lin2023}. 
%


% ? Neuer Punkt
However, the advent of the new student-centered approach based on artificial intelligence should not be viewed as a replacement for traditional teachers. Rather, 
the technology can assist teachers by relieving them of some of their tasks \cite{7742268} \cite{Lin2023} \cite{Tanveer2020}. Nevertheless, the integration of 
these new technologies will undoubtedly alter the role of teachers, potentially leading to a transformation in the approach to teaching 
\cite{Lin2023} \cite{9670009}.  
Furthermore, the technologies can address the shortage of teachers or educational opportunities \cite{cai2021review} \cite{Blanchard2015}.  
% ? bis hier sind es im findings 1900 zeichen 


% ? Challanges ab hier 
One of the most significant obstacles is the absence of the requisite infrastructure to effectively integrate the novel AI-driven technologies 
into the conventional pedagogical framework 
\cite{Lin2023}. 
%
The issue is particularly prevalent in developing countries, where access to and affordability of electricity, hardware, and data are significant obstacles
\cite{Tanveer2020}. 
%
Additionally, there is the challenge of ensuring that students and educators have the necessary skills to navigate the evolving landscape of digital technologies.
This could necessitate the provision of training for each individual 
\cite{Tanveer2020} \cite{10409934}.
%
Another issue arises in the context of data protection and security \cite{Lin2023}. The challenge here is that the personal data of learners must be collected and processed, 
while at the same time, security and data protection must be guaranteed \cite{Tanveer2020}.   
%
Furthermore, the potential for AI models to contain unintended biases, which may arise from the selection of training data, represents a significant challenge \cite{Lin2023}. 
Such biases could result in the algorithms discriminating against minorities \cite{angwin2022machine}, thereby undermining the fundamental objectives of ESD.
%
A further consideration is the potential for learners to become overly reliant on technology, which could result in a dependency on the technology itself. This could ultimately 
lead to learners being unable to perform without the technology \cite{Chin2010}.



%
\section{Discussions}
Although the articles were selected based on the criteria outlined in the Methods section, a systematic review process was not employed in this research. A more 
reliable set of results could have been achieved if the sources had been vetted using a systematic review approach, such as the one presented by Kitchenham 
\cite{kitchenham2007guidelines}.

Dazu kommmt das auch eine Umfangreicherer 



%
%
\section{Conclusions and Future Directions}


\newpage 

\printbibliography % im Falle der Nutzung von biblatex
\end{document}
